\documentclass[aps,prl,twocolumn,groupedaddress]{revtex4-2}

\usepackage{graphicx}
\usepackage{amssymb}
\usepackage{amsmath}
\usepackage{color}
\usepackage{hyperref}
\usepackage{babel}[spanish] % To write in spanish
\usepackage[utf8]{inputenc}
\setcounter{secnumdepth}{3}

\DeclareMathOperator*{\argmax}{arg\,max}
\DeclareMathOperator*{\argmin}{arg\,min}
\newcommand{\avrg}[1]{\left\langle #1 \right\rangle}
\newcommand{\nelta}{\bar{\delta}}

\begin{document}

\title{Ejemplo de artículo LaTex}

\author{Juan I. Perotti}
\email[]{juan.perotti@unc.edu.ar}
\affiliation{Instituto de Física Enrique Gaviola (IFEG-CONICET), Ciudad Universitaria, 5000 Córdoba, Argentina}
\affiliation{Facultad de Matemática, Astronomía, Física y Computación, Universidad Nacional de Córdoba, Ciudad Universitaria, 5000 Có|rdoba, Argentina}

\date{\today}

\begin{abstract}
Este es el resúmen al inicio del artículo.
\end{abstract}

\maketitle

\section{\label{intro}Introducción}

En los libros~\cite{verhulst1985nonlinear,wiggins2003introduction} podemos aprender sobre formas normales.

\subsection{\label{resultados}Resultados}

El atractor de Lorenz viene dado por la ecuación
\begin{eqnarray}
\label{eq1}
\dot{x}_1 &=& \sigma(x_2-x_1) \\
\dot{x}_2 &=& x_1(\rho-x_3)-x_2 \nonumber \\
\dot{x}_3 &=& x_1x_2-\beta x_3 \nonumber
\end{eqnarray}

\begin{figure}
\includegraphics*[scale=.9]{fig1.pdf}
\caption{
\label{fig1}
Aquí va la descripción de la figura.
}
\end{figure}

En la figura~\ref{fig1}, podemos apreciar el atractor de Lorenz generado a partir de la ODE en la ecuación~\ref{eq1}.
Esta figura fué extraída del primer libro mencionados en la sección~\ref{intro}.

Así incluimos direcciones web~\url{https://www.overleaf.com/learn/latex/Learn_LaTeX_in_30_minutes}.

\bibliography{references}

\onecolumngrid
\appendix

\section{Apéndice A}
\label{appA}

Los apéndices los incluimos en formato de una columna.

\end{document}
